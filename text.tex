\documentclass[a4paper, 12pt]{article}

\usepackage[ngerman]{babel} 
\usepackage[T1]{fontenc}
\usepackage{amsfonts} 
\usepackage{setspace}
\usepackage{amsmath}
\usepackage{amssymb}
\usepackage{titling}
\usepackage{hyperref}
\usepackage{csquotes} % for \textquote{}
% REMINDER: USE IEEEeqnarray* FOR ALINGMENTS%
\usepackage{IEEEtrantools}
\usepackage{stix}

\newcommand*{\puffer}{\text{ }\text{ }\text{ }\text{ }}
\newcommand*{\gedanke}{\textbf{-- }}
\newcommand*{\gap}{\text{ }}
\newcommand*{\setDef}{\gap|\gap}
% Hab länger gebraucht um zu realisieren, dass das ne gute Idee wäre
\newcommand*{\R}{\mathbb R}
\newcommand*{\grad}{\text{grad}}
\newcommand*{\J}{\textbf{J}}

% ¯\_(ツ)_/¯
\usepackage{tikz}
\newcommand{\shrug}[1][]{%
\begin{tikzpicture}[baseline,x=0.8\ht\strutbox,y=0.8\ht\strutbox,line width=0.125ex,#1]
\def\arm{(-2.5,0.95) to (-2,0.95) (-1.9,1) to (-1.5,0) (-1.35,0) to (-0.8,0)};
\draw \arm;
\draw[xscale=-1] \arm;
\def\headpart{(0.6,0) arc[start angle=-40, end angle=40,x radius=0.6,y radius=0.8]};
\draw \headpart;
\draw[xscale=-1] \headpart;
\def\eye{(-0.075,0.15) .. controls (0.02,0) .. (0.075,-0.15)};
\draw[shift={(-0.3,0.8)}] \eye;
\draw[shift={(0,0.85)}] \eye;
% draw mouth
\draw (-0.1,0.2) to [out=15,in=-100] (0.4,0.95); 
\end{tikzpicture}}


\pagestyle{plain}
\allowdisplaybreaks

\setlength{\droptitle}{-14em}
%\setlength{\jot}{12pt}

\title{\scshape Mathe C Klausurzettel}
\author{\scshape Henri Heyden\\\small stu240825}
\date{}

\begin{document}
\maketitle
\subsection*{Analysis}
\subsubsection*{Integrierbarkeit}
\text{\scshape Riemann Summe}\\
Sei \(f : [a,b] \rightarrow \R\), seien \((x, \xi)\) Partition und Stützstellen aus \([a,b]\).\\
Dann nennen wir \(R(f,x,\xi) := \sum_{i=1}^n (x_i - x_{i-1}) \cdot f(\xi_i)\) Riemann Summe. \\ \\
\text{\scshape Integrierbarkeit}\\
Wir nennen \(f\) integrierbar, wenn\\\(\exists R_0 \in \R \forall \epsilon > 0 \exists \delta > 0 \forall(x,\xi) \in \text{PS}(a,b,\delta): |R(f,x,\xi) - R_0| < \epsilon\) gilt. \\
Das Integral ist eindeutig, schreibe \(\int_{a}^{b} f\) oder \(\int f\) hierfür. \\
Wir schreiben auch \(\int f(x)\text dx := \int f\) \\
Ist \(f\) integrierbar, dann kann man das Integral mit einer Beliebigen Folge an \((x_n,\xi_n)_n\) finden wessen Feinheit den Limes 0 hat, sodass die Riemann-Summe konvergiert. \\
\(f\) ist genau dann integrierbar, wenn für alle 2 solcher Folgen ihre Differenz immer zu 0 konvergiert. \\ \\
\text{\scshape Stetig und Kompakt} \\
Eine Funktion \(f\): \\
\dots ist stetig in \(x \in \text{dom}(f)\), wenn alle Funktionslimetes zu \(x\) gleich sind.\\
\dots ist beschränkt, wenn ihre Domain eine obere und untere Schranke hat. \\
\dots ist abgeschlossen, wenn das komplement ihrer Domain offen ist. \\
\dots ist kompakt, wenn sie beschränkt und abgeschlossen ist. \\
Ist eine Funktion kompakt stetig, dann ist sie gleichmäßig stetig und somit integrierbar. \\ \\
\text{\scshape Abschätzungen} \\
Für \(f \le g\) gilt: \(\int f \le \int g\). \\
Es gilt: \((b-a) \cdot \inf(f) \le \int_{a}^{b}f \le (b-a) \cdot \sup(f)\)
\subsubsection*{Integrationstechniken}
\text{\scshape Hauptsatz der Differenzialrechnung} \\
Schreibe \([\phi]_u^v := \phi(v) - \phi(u)\) \\
Sei \(f,F: \Omega \rightarrow \R\) so, dass \(F' = f\) gilt. \\
Dann gilt: \(\int f = F(\sup(\Omega)) - F(\inf(\Omega)) = F(b) - F(a) = [F]_a^b\) für \(\Omega = [a,b]\).
\pagebreak \\
\text{\scshape Stammfunktionen} \\ \\
\begin{tabular}{ c | c | c | c }
    \hline
    Domain & \(f(x)\) & \(F(x)\) & args \\
    \hline
    \(\R\) & \(c\) & \(cx\) & \(c \in \R\) \\
    \(\R\) & \(\sum_{k=0}^{n}a_kx^k\) & \(\sum_{k=0}^{n}\frac{a_k}{k+1}x^{k+1}\) & \(a_0 \dots a_n \in \R\) \\
    \(\R_{> 0}\) & \(x^{-1}\) & \(\ln(x)\) & \\
    \(\R\) & \(b^x\) & \(\frac{b^x}{\ln(b)}\) & \(b \in \R_{>0} \setminus \{1\}\) \\
    \(\R_{> 0}\) & \(\log_b(x)\) & \(\frac{x\ln(x) - x}{\ln(b)}\) & \(b \in \R_{>0} \setminus \{1\}\) \\
    \(]-1, 1[\) & \(\frac{1}{\sqrt{1-x^2}}\) & arcsin\((x)\) & \\
    \(\R\) & \(\frac{1}{1+x^2}\) & arctan\((x)\) & \\
\end{tabular} \\ \\ \\
\text{\scshape Partielle Integration} \\
Für \(f,g : [a,b] \rightarrow \R\) gilt: \(\int_{a}^{b}fg' = [fg]_a^b - \int_{a}^{b}f'g\) \\ \\
\text{\scshape Substitution} \\
Für \(f\) stetig reel und \(\phi\) stetig differenzierbar reel mit \(u,v \in \text{dom}(\phi)\),\\
sodass \([u,v] \subseteq \text{dom}(\phi)\) und \(\phi^\rightarrow([u,v]) \subseteq \text{dom}(f)\) ist, gilt:
\[\int_{\phi(u)}^{\phi(v)}f = \int_{u}^{v}(f \circ \phi) \cdot \phi'\] \\

\subsection*{Analytische Grundstrukturen}
\subsection*{Differentation im Mehrdimensionalen}
\subsection*{Stochastik}
\end{document}