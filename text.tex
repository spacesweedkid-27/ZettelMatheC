% finally done it
\documentclass[a4paper, 12pt]{article}

\usepackage[ngerman]{babel} 
\usepackage{setspace}
\usepackage{amsfonts}
\usepackage{amsmath}
\usepackage{bbm}
\usepackage{mathrsfs}   % for \mathscr
\usepackage{titling}
\usepackage{hyperref}
\usepackage{csquotes} % for \textquote{}
% REMINDER: USE IEEEeqnarray* FOR ALINGMENTS%
\usepackage{IEEEtrantools}

\usepackage{stix}

\newcommand*{\puffer}{\text{ }\text{ }\text{ }\text{ }}
\newcommand*{\gedanke}{\textbf{-- }}
\newcommand*{\gap}{\text{ }}
\newcommand*{\setDef}{\gap|\gap}
% Hab länger gebraucht um zu realisieren, dass das ne gute Idee wäre
\newcommand*{\R}{\mathbbm R}
\newcommand*{\N}{\mathbbm N}
\newcommand*{\Z}{\mathbbm Z}
\newcommand*{\grad}{\text{grad}}
\newcommand*{\J}{\textbf{J}}
\newcommand*{\He}{\textbf{H}}
\newcommand*{\smi}{\text{-}}
\newcommand*{\Co}{\mathcal C^1}
\newcommand*{\Ct}{\mathcal C^2}
\newcommand*{\todo}{\textit{\dots TODO \dots}}
\newcommand*{\LMAX}{\text{\scshape Lmax}}
\newcommand*{\LMIN}{\text{\scshape Lmin}}
\newcommand*{\QDp}{{:}\text{ }}
\newcommand*{\E}{\mathcal E}
\newcommand*{\Pro}{\mathbbm P}
\newcommand*{\pro}{\mathcal P}
\newcommand*{\Od}{\text{Od}}
\newcommand*{\supp}{\text{supp}}
\newcommand*{\img}{\text{img}}
\newcommand*{\dom}{\text{dom}}
\newcommand{\subsubsubsection}[1]{\textsc{#1} \\} 

% ¯\_(ツ)_/¯
\usepackage{tikz}
\newcommand{\shrug}[1][]{%
\begin{tikzpicture}[baseline,x=0.8\ht\strutbox,y=0.8\ht\strutbox,line width=0.125ex,#1]
\def\arm{(-2.5,0.95) to (-2,0.95) (-1.9,1) to (-1.5,0) (-1.35,0) to (-0.8,0)};
\draw \arm;
\draw[xscale=-1] \arm;
\def\headpart{(0.6,0) arc[start angle=-40, end angle=40,x radius=0.6,y radius=0.8]};
\draw \headpart;
\draw[xscale=-1] \headpart;
\def\eye{(-0.075,0.15) .. controls (0.02,0) .. (0.075,-0.15)};
\draw[shift={(-0.3,0.8)}] \eye;
\draw[shift={(0,0.85)}] \eye;
% draw mouth
\draw (-0.1,0.2) to [out=15,in=-100] (0.4,0.95); 
\end{tikzpicture}}


\pagestyle{plain}
\allowdisplaybreaks
\setlength{\hoffset}{-1in}
\setlength{\voffset}{-1in}
\setlength{\textheight}{26cm} % DIN A4 ~ 30cm mir ist der obere rand egal
\setlength{\textwidth}{18cm}  % DIN A4 = 21cm, 18 reichen mir.
\usepackage[ngerman]{babel} 

\begin{document}
\section*{Mathe C Klausurzettel}
\textbf{Henri Heyden \gedanke stu240825}
\subsection*{Analysis}
\subsubsection*{Integrierbarkeit}
\subsubsubsection{\scshape Riemann Summe}{
    Sei \(f : [a,b] \longrightarrow \R\), seien \((x, \xi)\) Partition und Stützstellen aus \([a,b]\).\\
    Dann nennen wir \(R(f,x,\xi) := \sum_{i=1}^n (x_i - x_{i-1}) \cdot f(\xi_i)\) Riemann Summe.
}
\subsubsubsection{\scshape Integrierbarkeit}{
    Wir nennen \(f\) integrierbar, wenn\\\(\exists R_0 \in \R \forall \epsilon > 0 \exists \delta > 0 \forall(x,\xi) \in \text{PS}(a,b,\delta): |R(f,x,\xi) - R_0| < \epsilon\) gilt. \\
    Das Integral ist eindeutig, schreibe \(\int_{a}^{b} f\) oder \(\int f\) hierfür. \\
    Wir schreiben auch \(\int f(x)\text dx := \int f\) \\
    Ist \(f\) integrierbar, dann kann man das Integral mit einer Beliebigen Folge an \((x_n,\xi_n)_n\) finden wessen Feinheit den Limes 0 hat, sodass die Riemann-Summe konvergiert. \\
    \(f\) ist genau dann integrierbar, wenn für alle 2 solcher Folgen ihre Differenz immer zu 0 konvergiert.
}
\subsubsubsection{\scshape Stetig und Kompakt}{
    Eine Funktion \(f\): \\
    \dots ist stetig in \(x \in \text{dom}(f)\), wenn alle Funktionslimetes zu \(x\) gleich sind.\\
    \dots ist beschränkt, wenn ihre Domain eine obere und untere Schranke hat. \\
    \dots ist abgeschlossen, wenn das komplement ihrer Domain offen ist. \\
    \dots ist kompakt, wenn sie beschränkt und abgeschlossen ist. \\
    Ist eine Funktion kompakt stetig, dann ist sie gleichmäßig stetig und somit integrierbar.\\
    Aus Lipschitzstetigkeit (\(\exists L\in \R\QDp \forall x_1,x_2\in \dom(f)\QDp |f(x_1)-f(x_2)| \le L \cdot |x_1 - x_2|\)) folgt auch gleichmäßige Stetigkeit.
}
\textsc{\underbar{Abschätzungen}}\\
    Für \(f \le g\) gilt: \(\int f \le \int g\). \\
    Es gilt: \((b-a) \cdot \inf(f) \le \int_{a}^{b}f \le (b-a) \cdot \sup(f)\)
\subsubsection*{Integrationstechniken}
\subsubsubsection{Hauptsatz der Differenzialrechnung}{
    Schreibe \([\phi]_u^v := \phi(v) - \phi(u)\) \\
    Sei \(f,F: \Omega \longrightarrow \R\) so, dass \(F' = f\) gilt. \\
    Dann gilt: \(\int f = F(\sup(\Omega)) - F(\inf(\Omega)) = F(b) - F(a) = [F]_a^b\) für \(\Omega = [a,b]\).
}
\subsubsubsection{Partielle Integration}{
    Für \(f,g : [a,b] \longrightarrow \R\) gilt: \(\int_{a}^{b}fg' = [fg]_a^b - \int_{a}^{b}f'g\)
}
\textsc{\underbar{Substitution}}\\
    Für \(f\) stetig reel und \(\phi\) stetig differenzierbar reel mit \(u,v \in \text{dom}(\phi)\),\\
    sodass \([u,v] \subseteq \text{dom}(\phi)\) und \(\phi^\rightarrow([u,v]) \subseteq \text{dom}(f)\) ist, gilt:
    \[\int_{\phi(u)}^{\phi(v)}f = \int_{u}^{v}(f \circ \phi) \cdot \phi'\] \\
\text{\scshape \underbar{Stammfunktionen}} \\ \\
\begin{tabular}{ c | c | c | c }
    \hline
    Domain & \(f(x)\) & \(F(x)\) & args \\
    \hline
    \(\R\) & \(c\) & \(cx\) & \(c \in \R\) \\
    \(\R\) & \(\sum_{k=0}^{n}a_kx^k\) & \(\sum_{k=0}^{n}\frac{a_k}{k+1}x^{k+1}\) & \(a_0 \dots a_n \in \R\) \\
    \(\R_{> 0}\) & \(x^{-1}\) & \(\ln(x)\) & \\
    \(\R\) & \(b^x\) & \(\frac{b^x}{\ln(b)}\) & \(b \in \R_{>0} \setminus \{1\}\) \\
    \(\R_{> 0}\) & \(\log_b(x)\) & \(\frac{x\ln(x) - x}{\ln(b)}\) & \(b \in \R_{>0} \setminus \{1\}\) \\
    \(]\smi 1, 1[\) & \(\frac{1}{\sqrt{1-x^2}}\) & arcsin\((x)\) & \\
    \(\R\) & \(\frac{1}{1+x^2}\) & arctan\((x)\) & 
\end{tabular} \\
\textsc{\underbar{Logarithmus und Exponentialfunktion}} \\
    Es gilt \(\ddx\log_a(x) = \ddx\frac{\ln(x)}{\ln(a)} = \frac{1}{x \cdot \ln(a)}\) und \(\ddx a^x = e^{\ln(a) \cdot x} = \ln(a) \cdot e^{\ln(a) \cdot x} = \ln(a) \cdot a^x\)
\subsubsection*{Uneigentliche Integrale}
\text{\scshape \underbar{Integral über \(x^{-\alpha}\)}} \\
1) \(\forall \alpha > 1: \int_{1}^{+\infty} x^{-\alpha} \text d x = \frac{1}{\alpha - 1}\). \puffer Für \(\alpha \in {]0{,}1]}\) divergiert das Integral. \\
2) \(\forall \alpha \in ]0,1[: \int_{0}^{1}x^{-\alpha} \text d x = \frac{1}{1 - \alpha}\). \puffer Für \(\alpha \ge 1\) divergiert das Integral. \\ \\
\text{\scshape \underbar{Vergleichskriterium}} \\
Seien \(f,g: [a,b[ \longrightarrow \R\) integrierbar über alle Teilintervalle und \(|f| \le g\). \\
Konvergiert das uneigentliche Integral über \(g\), dann konvergiert das uneigentliche Integral über \(f\). \\ \\
\text{\scshape \underbar{Integralkriterium}} \\
Warnung nicht im Skript, trotzdem leichtes Argument. \\
Sei \(p \in \Z\) und \(f : [p, \infty[ \longrightarrow \R_{> 0}\) monoton fallend. \\
Dann ist \(f\) integrierbar genau dann, wenn \(\sum_{n = p}^{\infty} f(n)\) konvergiert. \\
Zur Motivation betrachte \(\sum_{n = p}^{\infty} f(n)\) als obere Schranke und Annäherung des Integrals mittels Riemann Summe mit gleichmäßiger Feinheit 1.
\subsection*{Analytische Grundstrukturen}
\subsubsection*{Metrische Räume}
\text{\scshape \underbar{Metrik}} \\
Eine Metrik ist eine Funktion \(d : M^2 \longrightarrow \R_{\ge 0}\) mit folgenden Eigentschaften:\\
1) \(d(x,y) = 0 \Longleftrightarrow x = y\) \puffer \puffer \gap \gap (positive Definitheit) \\
2) \(d(x,y) = d(y,x)\) \puffer \puffer \puffer \puffer \gap \gap \gap (Symmetrie) \\
3) \(d(x,y) \le d(x,z) + d(z,y)\) \puffer (Dreiecksungleichung) \\ \\
\text{\scshape \underbar{Eigenschaften der Metrik}} \\
Für Metrische Räume gelten alle analytischen Gesetze und Sätze aus MatheB nur mit jeder Metrik nicht nur der Betragsmetrik.
Somit sind Begriffe wir Stetigkeit, Limes, Kompaktheit, Funktionslimes etc. äquivalent.
\subsubsection*{Normierte Räume}
\text{\scshape \underbar{Norm}} \\
Sei \(V\) Vektorraum, dann ist \(||\cdot||\) Norm auf \(V\), wenn folgende Gesetze gelten:\\
1) \(||v|| = 0 \Longleftrightarrow v = 0\) \\
2) \(||\lambda|| = |\lambda| \cdot ||v||\) \\
3) \(||u+v|| \le ||u|| + ||v||\) \\ \\
\text{\scshape \underbar{Beispielnormen über \(\R^n\)}} \\
Sei \(k \in \N_{> 0}\), dann ist \(||\cdot||_k : \R^n \longrightarrow \R, v \mapsto \sqrt[k]{\sum_{i=1}^{n}|v_i|}\) Norm \\
Die Funktion \(||\cdot||_\infty : \R^n \longrightarrow \R, v \mapsto \max_{i \in [n]}|v_i|\) ist Norm \\ \\
\text{\scshape \underbar{Eigenschaften der Norm}} \\
Es gilt \(||-v|| = ||v||\) \\
Außerdem ist \(V^2 \longrightarrow \R_{\ge 0}, (u,v) \mapsto ||u-v||\) Metrik. Somit lassen sich analytische Grundbegriffe für Vektorräume komposieren und alle Eigentschaften der Metrischen Räume für jene Metrik anwenden. \\
Normen sind stetig. Also \(||\cdot||\) ist eine stetige Funktion. \\ \\
\text{\scshape \underbar{Äquivalenz von Normen}} \\
Seien \((V, ||\cdot||)\) und \((V, ||\cdot||')\) normierte Räume. Wir nennen \(||\cdot||\) und \(||\cdot||'\) äquivalent, wenn folgendes gilt: \(\exists \alpha,\beta > 0: \forall v \in V: \alpha ||v|| \le ||v||' \le \beta||v||\) \\
Umgebungen über äquivalente Normen sind äquivalent, genau wie Konvergenz mit den gleichen Limetes, Stetigkeiten und Kompaktheiten. \\
Auf endlich dimensionalen Vektorräumen sind alle Normen äquivalent.
\subsection*{Differentation im Mehrdimensionalen}
Es seien \(V, W\) Vektorräume über \(\R\), \(\Omega \subseteq V\).
\subsubsection*{Differenzierbarkeit}
\text{\scshape \underbar{Definiton der Differenzierbarkeit}} \\
Wir nennen \(f : \Omega \longrightarrow W\) differenzierbar in \(v\), wenn \[\exists \phi \in \text L(V,W): \lim_{\tilde v \longrightarrow v} \frac{||f(\tilde v) - (f(v) + \phi(\tilde v - v))||_W}{||\tilde v - v||_V} = 0_\R\]
gilt. Dann nennen wir \(\phi\) Ableitung von \(f\) und schreiben \(D(f) := \phi\). Offenbar da \(\phi\) linear stetig ist, existiert eine Matrix für \(\phi\). Folgende Definiton ist äquivalent: \[\exists \phi \in \text L(V,W): \lim_{h \longrightarrow 0_V} \frac{||f(v + h) - (f(v) + \phi(h))||_W}{||h||_V} = 0_\R\]
Gilt \(V = W = \R\), dann ist die Definiton äquivalent zur Differenzierbarkeit im bekannten Sinne.
\subsubsection*{Ableitungsregeln}
Ist \(f\) differenzierbar, dann sind alle Komponentenfunktionen von \(f\) \((V,\R)\)-differenzierbar. \\ \\
\text{\scshape \underbar{Kombinationssätze}} \\
Für \(\lambda : \Omega \longrightarrow \R\), \(f,g : \Omega \longrightarrow V\) differenzierbar gelten folgende Regeln:\\
1) \(D(f+g) = D(f) + D(g)\) \\
2) \(D(\lambda f) = \lambda D(f)\) \\
3) \(D\left(\frac{1}{\lambda}\right) = \frac{D(\lambda)}{\lambda^2}\) \\
4) \(D\left(\frac{f}{\lambda}\right) = \frac{\lambda D(f) - fD(\lambda)}{\lambda^2}\)
\subsubsection*{Partielle Ableitungen}
Seien \(\Omega \subseteq \R^n, f : \Omega \longrightarrow \R^d\), \(i \in [n], j \in [d], v \in \Omega\). \\ \\
\text{\scshape \underbar{Partielle Differenzierbarkeit}} \\
Wir nennen \(f_j\) partiell differenzierbar in \(v\) nach der i-ten Variable, wenn \(f_{j,v,i}\) in \(v_i\) differenzierbar ist.\\
Wir definieren \(\partial_i f_j(v) := f_{j,v,i}'(v_i)\). \\ \\
\text{\scshape \underbar{Beispiel}} \\
Sei \(f: \R \times \R_{\ge 0} \longrightarrow \R^4, (x,y) \mapsto \left(x+y, xy, x^2+3y-6, -2\sqrt{y}\right)\). \\
Dann gilt:\\
\(\partial_1f_2(x,y) = y\), \puffer
\(\partial_1f_3(x,y) = 2x\), \puffer
\(\partial_1f_4(x,y) = 0\), \\
\(\partial_2f_1(x,y) = 1\), \puffer
\(\partial_2f_3(x,y) = 3\) \\ \\
\text{\scshape \underbar{Gradient und Jacobi-Matrix}} \\
Wir definieren den Gradienten von \(f_j\) als \(\grad_{f_j}(v) := (\partial_1f_j(v), \dots, \partial_nf_j(v))\). \\
Des Weiteren definieren wir die Jacobi-Matrix von \(f\):\[\J_f(v) := \begin{bmatrix}
    \partial_1f_1(v) & \hdots & \partial_nf_1(v) \\
    \vdots & & \vdots \\
    \partial_1f_d(v) & \hdots & \partial_nf_d(v)
\end{bmatrix}\] Somit gilt: \[\J_f(v) = \begin{bmatrix}
    \grad_{f_1}(v) \\
    \vdots \\
    \grad_{f_d}(v)
\end{bmatrix}\]
Ist \(f\) differenzierbar in \(v\), dann ist \(\J_f(v)\) Ableitung von \(f\), aber Achtung: Vollständige partielle Differenzierbarkeit impliziert NICHT totale Differenzierbarkeit! \\ \\
\text{\scshape \underbar{Totale Differenzierbarkeit aus Partieller Differenzierbarkeit}} \\
Ist \(f\) vollständig partiell differenzierbar und alle Ableitung sind stetig, dann ist \(f\) total differenzierbar. \\ \\
\text{\scshape \underbar{\(\Co\)-Funktionen und \(\Ct\)-Funktionen}} \\
Wir nennen \(f: \Omega \longrightarrow \R^d\) eine \(\Co\)-Funktion, wenn sie vollständig stetig partiell differenzierbar ist. Somit ist also eine \(\Co\)-Funktion total differenzierbar.
\(f\) ist \(\Co\)-Funktion, wenn all ihre Komponentenfunktionen aus Kombinationen von Projektionen stetiger differenzierbaren eindimensionalen Funktionen bestehen. \\
Wir nennen \(f: \Omega \longrightarrow \R\) eine \(\Ct\)-Funktion, wenn sie eine \(\Co\)-Funktion ist, und \(\forall i \in [n]: \partial_if\) stetig partiell differenzierbar ist. \\
Dies ist äquivalent dazu, dass \(\grad_f\) eine \(\Co\)-Funktion ist. \\Die partielle
Ableitung von \(\partial_i f\) nach der \(j\)-ten Variable bezeichnen wir mit \(\partial_j\partial_if\). \\ \\
\text{\scshape \underbar{Hesse-Matrix}} \\
Sei \(f: \Omega \longrightarrow \R\) eine \(\Ct\)-Funktion, wir definieren die Hesse-Matrix von \(f\) als: \[\He_f := \begin{bmatrix}
    \partial_1\partial_1f(v) & \hdots & \partial_1\partial_nf(v) \\
    \vdots & & \vdots \\
    \partial_n\partial_1f(v) & \hdots & \partial_n\partial_nf(v)
\end{bmatrix}\] offenbar gilt: \(\He_f(v) = (\J_{\grad_f}(v))^{\top}\), denn die Spalten der Hesse-Matrix sind die
Gradienten der Komponentenfunktionen von \(\grad_f\).
Die Transposition ist auch nicht nötig, da die Hesse-Matrix symmetrisch ist. Das heißt es gilt: \(\forall i,j\in[n]{:}\text{ } \partial_i\partial_jf = \partial_j\partial_if\).
\subsubsection*{Lokale Extremstellen}
\text{\scshape \underbar{Definiton}} \\
Wir nennen \(v \in \Omega\) eine lokale Maximumstelle/Minimumstelle von \(f\), wenn gilt:\\
\(\exists U \in \mathcal U(v){:} \forall u \in U{:} f(u) \le f(v)\) bzw. \(f(u) \ge f(v)\). \\
Wir bezeichnen die Menge der lokalen Maximumstellen \(\LMAX(f)\) und die Menge der lokalen Minimumstellen \(\LMIN(f)\). Die Vereinigung beider Mengen ist \(\text{\scshape Lext}(f)\). \\ \\
\text{\scshape \underbar{Definitheit}} \\
Sei \(A\in\R^{n\times n}\) symmetrisch, dann nennen wir \(A\) \dots \\
\dots positiv definit, wenn \(\forall k \in [n]: \det(A_k) > 0\) \\
\dots negativ definit, wenn \(\forall k \in [n]: \det(A_k) < 0\) \\
\dots indifinit, sonst. \\ \\
\text{\scshape \underbar{Definitheit der Hesse-Matrix}} \\
Definiere die Menge der kritischen Stellen von \(f\) als \(K(f) := \grad_f^\leftarrow(0_{\R^n})\). \\
Es gilt für alle \(v \in K(f)\): \\
1) \(\He_f(v)\) positiv definit \(\Longrightarrow v \in \LMIN(f)\) \\
2) \(\He_f(v)\) negativ definit \(\Longrightarrow v \in \LMAX(f)\) \\
3) \(\He_f(v)\) indifinit \(\Longrightarrow v \not\in \text{\scshape Lext}(f)\) \\ \\
\text{\scshape \underbar{Definitheit einer \(2\times 2\)-Matrix}} \\
Definiere \(A := \begin{bmatrix}
    a & b \\ b & c
\end{bmatrix}\) für \(a,b,c \in \R\). Dann gilt:\\
1) \(\det(A) = ac-b^2\) \\
2) \(A\) positiv definit \(\Longleftrightarrow a > 0 \wedge \det(A) > 0\). \\
3) \(A\) negativ definit \(\Longleftrightarrow a < 0 \wedge \det(A) > 0\). \\
4) \(A\) indefinit \(\Longleftrightarrow \det(A) < 0\)
\subsection*{Stochastik}
\subsubsection*{Mathematische Modellierung des Zufalls}
\text{\scshape \underbar{\(\sigma\)-Algebra, Eventraum}} \\
Sei \(\Omega \ne \emptyset\) Menge und \(\mathcal E \subseteq \mathcal P(\Omega)\). Für \(A \in \mathcal P(\Omega)\) definieren wir \(A^c := \Omega \setminus A\). \\
Wir nennen \(\mathcal E\) eine \(\sigma\)-Algebra über \(\Omega\), wenn gilt:\\
1) \(\Omega \in \mathcal E\) \\
2) \(\forall A \in \mathcal E{:} A^c \in \mathcal E\) \\
3) \(\forall (A_i)_i \in \mathcal S(\mathcal E){:} \bigcup_i A_i \in \mathcal E\)\\
Wir nennen \((\Omega, \mathcal E)\) \textbf{Eventraum} und \(\Omega\) \textbf{Ergebnisraum}. \\
Die Elemente von \(\mathcal E\) nennen wir \textbf{Event} und die Elemente von \(\Omega\) \textbf{Ergebnis}. \\ Einen E'Raum nennen wir diskret, wenn \(\Omega\) abzählbar ist \\ \\
\text{\scshape \underbar{Beispiele}} \\
Für \(\Omega\) ist \(\{\emptyset, \Omega\}\) \(\sigma\)-Algebra über \(\Omega\). Dann ist \((\Omega, \{\emptyset, \Omega\})\) Eventraum, genau wie \((\Omega, \mathcal P(\Omega))\) \\ \\
\text{\scshape \underbar{Wahrscheinlichkeitsmaß}} \\
Sei \((\Omega, \mathcal E)\) Eventraum. Sei eine Funktion \({\mathbb P}:{\mathcal E}{\longrightarrow}{[0,1]}\), dann nennen wir \(\mathbb P\) Wahrscheinlichkeitsmaß auf \((\Omega, \mathcal E)\), wenn gilt:\\
1) \(\mathbb P(\Omega) = 1\) \\
2) \(\forall(A_i)_i \in \mathcal S(\mathcal E), \bigcap_iA_i = \emptyset\QDp \mathbb P\left(\bigcup_iA_i\right) = \sum_{i}\mathbb P(A_i)\) \\
Das Tripel \((\Omega, \mathcal E, \mathbb P)\) nennen wir Wahrscheinlichkeitsraum. \\ \\
\subsubsubsection{Grundlegende Eigenschaften der Wahrscheinlichkeit}{
    1) \(\Pro(\emptyset) = 0\) \\
    2) \(\forall A,B \in \E, A{\cap} B = \emptyset\QDp \Pro(A{\cup} B) = \Pro(A) + \Pro(B)\) \\
    3) \(\forall A,B \in \E, A \subseteq B\QDp \Pro(B{\setminus} A) = \Pro(B) - \Pro(A)\) \\
    4) \(\forall A \in \E: \Pro(A^c) = 1-\Pro(A)\) \\
    5) \(\forall A,B \in \E \QDp \Pro(A{\cup} B) = \Pro(A) + \Pro(B) - \Pro(A{\cap}B)\) 
}
\subsubsubsection{Uniformes W'Maß}{
    Nun beachte, dass wenn \(\Pro(A)\) und \(\Pro(B)\) bekannt sind für \(A{\cap}B = \emptyset\), dass dann auch \(\Pro(A{\cup}B)\) berechnet werden kann. Somit lassen sich Wahrscheinlichkeitsmaße konstruieren, wenn \(\forall \omega \in \Omega: \Pro(\omega)\) bekannt ist. \\
Eines solcher W'Maße ist das Uniforme mit \(\Pro(\omega) := \frac{1}{|\Omega|}\). Dann gilt: \(\Pro(A) = \frac{|A|}{|\Omega|}\).
}
\subsubsubsection{Konditionierung}{
    Seien \(A,B \in \E\) so, dass \(\Pro(B) > 0\). Wir definieren die W'keit von A konditioniert auf B, oder W'keit von A gegeben B als \(\Pro(A{|}B) := \frac{\Pro(A{\cap}B)}{\Pro(B)}\).
}
\subsubsubsection{\scshape Bayes}{
    Für \(A,B \in \E\) mit \(\Pro(A), \Pro(B) > 0\) gilt:
    \(\Pro(A{|}B) = \frac{\Pro(A)}{\Pro(B)} \cdot \Pro(B{|}A)\)
}
\subsubsubsection{Odds}{
    Wir definieren für \(A,B \in \E\) mit \(\Pro(B) > 0\) die Odds von A konditioniert auf B, bzw die Odds von A als:
    \[\Od(A{|}B) := \begin{cases}
        \frac{\Pro(A{|}B)}{\Pro(A^c{|}B)} = \frac{\Pro(A{\cap}B)}{\Pro(A^c{\cap}B)} & \Pro(A{|}B < 1) \\ +\infty & \Pro(A{|}B = 1)
    \end{cases}\]
    \[\Od(A) := \begin{cases}
        \frac{\Pro(A)}{\Pro(A^c)} & \Pro(A < 1) \\ +\infty & \Pro(A = 1)
    \end{cases} \puffer\puffer\puffer\puffer\puffer\puffer\gap\gap\]
    Folgende Umrechnungsregeln gelten: \\
    \(o := \frac{p}{1-p}\) sind die odds für \(p\) Wahrscheinlichkeit. \\
    \(p := \frac{o}{o+1}\) ist die Wahrscheinlichkeit für \(o\) odds.
}
\textsc{\underbar{Unabhängigkeit}}\\
    Sei \((A_i)_{i \in I}\) eine endliche Familie von Events. Wir nennen diese unabhängig, wenn gilt:\\ \(\forall I'\subseteq I\QDp \Pro\left(\bigcap_{i \in I'}A_i\right) = \prod_{i \in I'}\Pro(A_i)\)
\subsubsection*{Zufallsvariablen über diskreten Wahrscheinlichkeitsräumen}
\subsubsubsection{Zufallsvariablen}{
    Sei \((\Omega, \pro(\Omega), \Pro)\) diskreter W'Raum. \\
    Wir nennen alle Funktionen \(X\QDp \Omega {\longrightarrow} \R\) Zufallsvariablen.
}
\subsubsubsection{Events über Zufallsvariablen}{
    Sei \(\phi\) ein logischer Ausdruck der Form: \textquote{\({X \in U} / {X \le u} / {X = u}\)}, dann bezeichnen wir damit das Event, für das für \(X(\omega)\) der Ausdruck gilt für alle \(\omega \in \Omega\). Somit lassen sich Events mit ZV elegant konstruieren, und es lassen sich Wahrscheinlichkeiten wie \(\Pro(X = u) = \Pro(\{{\omega \in \Omega} \mid {X(\omega) = u}\})\) berechnen.
}
\textsc{\underbar{Gesetze der neuen Events}} \\
    1) \(\Pro(\bot \mid \beta) = 0 \wedge \Pro(\top|\beta) = 1\) \\
    2) \(\Pro(\neg\phi \mid \beta) = 1 - \Pro(\phi|\beta)\) \\
    3) \(\Pro(\phi {\vee} \psi \mid \beta) = \Pro(\phi| \beta) + \Pro(\psi| \beta) - \Pro(\phi {\wedge} \psi| \beta)\)
\subsubsection*{Verteilungen über Zufallsvariablen}
\subsubsubsection{Verteilung}{
    Man sieht, dass die Funktion \(\Pro_X\QDp \mathcal P(\R) \longrightarrow [0,1], U \mapsto \Pro(X {\in} U)\) ein W'maß für den Eventraum \((\R, \mathcal P(\R))\) ergibt gegeben der ZV \(X\) und dem diskretem Eventraum \((\Omega, \mathcal P(\Omega), \Pro)\). \\ Diese Funktion nennen wir Verteilung von \(X\).
}
\subsubsubsection{Träger}{
    Wir definieren den \textbf{Träger} von \(X\) als: \(\supp(X) := \{{u \in \R} \mid {\Pro(X=u) > 0}\}\). \\
    Offenbar ist \(X^\rightarrow(\Omega) = \img(X)\) abzählbar (da \(\Omega\) abzählbar ist), und es gilt: \(\supp(X) \subseteq \img(X)\), also ist der Träger einer ZV über diskreten W'räumen abzählbar. \\
    Somit schreibt man auch \(\Pro(X {\in} U) = \Pro(X {\in} {U \cap \supp(X)}) = \sum_{u \in U \cap \supp(X)}\Pro(X {=} u) =: \sum_{u \in U}\Pro(X {=} u)\).
}
\subsubsubsection{Probability Mass Function (PMF)}{
    Wir nennen \(\rho_X \QDp \R \longrightarrow [0,1],\gap u \mapsto \Pro(X = u)\) die PMF von \(X\).
    }
\subsubsubsection{Tupel und Familien von ZV}{
    Wir definieren eine Familie von ZV als: \(I \subseteq \N, X := \text{\huge\(\times\)}_{i \in I}X_i\), dann ist \(X\) Familie über alle ZV \(X_i \in M_i \in \mathcal P(M)^I\) mit \(M\) Menge an ZV. Ein Tupel einer ZV ist ein Spezialfall dessen, und ZVar, dass\\
    \(\exists n \in \N: I = [n]\) gilt. Wir schreiben dann \(X := (X_1 \cdot X_n)\) für Tupel und \((X_i)_{i\in I}\) für die Familie. \\
    Wir definieren somit folgende Events: \(X \in U :\Longleftrightarrow \forall i \in I: X_i \in U_i\) und \(X = u :\Longleftrightarrow \forall i \in I: X_i = u_i\). Die Verteilung, der Träger und die PMF werden gleich definiert.
}
\subsubsubsection{Uniforme Verteilung}{
Für eine Familie von ZV oder einer ZV \(X\) sagen wir, dass \(X\) uniform verteilt ist,\\
wenn \(\forall u \in M: \Pro(X = u) = \frac{1}{|M|}\) gilt. Folgendes ist die PMF einer Uniformen Verteilung:\\
\(\rho_{\Unif(M)}\QDp \R^I \longrightarrow [0,1], u \mapsto \begin{cases}
    \frac{1}{|M|} & u \in M\\
    0 & \text{sonst}
\end{cases}\)
}
\subsubsubsection{Geometrische Verteilung}{
    Sei \(p \in ]0,1[\), wir nennen eine ZV \(X\) geometrisch verteilt, wenn gilt: \(\Pro(X {=} k) = (1-p)^kp\)
}
\subsubsubsection{Poisson-Verteilung}{
    Sei \(\lambda > 0\), wir nennen eine ZV \(X\) poisson-verteilt mit EW \(\lambda\), wenn gilt: \(\Pro(X {=} k) = \frac{\lambda^k}{k!}e^{-\lambda}\)
}
\subsubsubsection{Marginalisierung Beispiel}{
    Sei \((X_1, X_2) \sim \Unif(\{(0,1), (0,2), (2,1)\})\). Dann gilt:\\
    \(\Pro(X_1 = 0)\\ = \Pro((X_1 = 0) \wedge (X_2 = 1 \vee X_2 = 2))\\ = \Pro((X_1,X_2) = (0,1) \vee (X_1,X_2) = (0,2))\\ = \Pro((X_1,X_2) = (0,1)) + \Pro((X_1,X_2) = (0,2))\\ = \frac{2}{3}\)
}
\subsubsubsection{Bernoulliverteilung}{
    Sei \(p \in [0,1]\) für eine ZV X sagen wir, dass sie Bernoulli-verteilt ist, wenn gilt:\\
    \(\Pro(X {=} 1) = p\) und \(\Pro(X {=} 0) = 1{-}p\), dann ist die PMF: \[\rho_{\Ber(p)} \QDp \R \longrightarrow [0,1], u \mapsto \begin{cases}
        p & u=1 \\
        1-p & u=0\\
        0 & \text{sonst}
    \end{cases}\]
}
\textsc{\underbar{Unabhängigkeit}}
    Wir nennen eine endliche Familie von ZV \(X = (X_i)_{i\in I}\) unabhängig, wenn gilt:
    \[\forall(U_i)_{i\in I} \in \mathcal P(\R)^I\QDp \Pro\left(\bigwedge_{i \in I}(X_i \in U_i)\right) = \prod_{i\in I}\Pro(X_i \in U_i)\]
    Wir sagen \((X_i)_{i \in I} \sim \mathcal D\) i.d.d. für eine Verteilung \(\mathcal D\), wenn wir meinen, dass alle \(X_i\) unabhängig aber gleich verteilt sind. \pagebreak \\
\textsc{\underbar{Summe unabhängiger Bernoulli-ZV}}\\
    Sei \(p \in [0,1]\) und \(X = (X_1, \dots, X_n) \sim \Ber(p)\) i.i.d., und definiere \(S := \sum_{i=1}^{n}X_i\), dann gilt für \(k \in [n]\):
    \[\Pro(S {=} k) = \binom{n}{k}p^k(1-p)^{n-k}\]
    \\ \\
\textsc{\underbar{Binominalverteilung}} \\
    Sei \(p \in [0,1]\) und \(X = (X_1, \dots, X_n) \sim \Ber(p)\) i.i.d., dann nennen wir \(X\) binomialverteilt und schreiben:
    \[(X_1, \dots, X_n) \sim \Ber(p) \text{ i.i.d.} \Longrightarrow \sum_{i=1}^{n}X_i \sim \Bin(n,p)\]
\subsubsection*{Erwartungswert und Varianz von ZV über diskreten W'räumen}
\subsubsubsection{Erwartungswert}{
    Sei \(X\) ein ZV. \(X\) hat einen Erwartungswert, genau dann, wenn \(\Ew := \sum_{u \in \R}u \cdot \Pro(X {=} u)\) definiert ist bzw. summierbar ist für \(|u|\) statt \(u\). Offenbar gilt: \(\Ew(X) = \sum_{\omega \in \Omega}X(\omega) \cdot \Pro(\omega)\).
}
\subsubsubsection{Erwartungswerte von Verteilungen}{
    Es gilt: \(\Ew(\Ber(p)) = p\), \(\Ew(\Bin(n,p)) = n\cdot p\), \(\Ew(\Pois(\lambda)) = \lambda\) und \(\Ew(\Geo(p)) = \frac{1-p}{p}\). Des Weiteren gilt: \(|\Ew(X)| \le \Ew(|X|)\)
}
\subsubsubsection{Law of the unconscious statistician (LOTUS)}{
    Sei \(X\) ZV oder endliche Familie davon und sei \(f\QDp \supp(X) \longrightarrow \R\). Dann hat die ZV \(f \circ X\) einen EW genau dann, wenn \(\sum_{u \in R^I}|f(u)| \cdot \Pro(X = u) < +\infty\) gilt. \\
    Des Weiteren lässt sich der EW von \(f \circ X\) so berechnen: \(\Ew(f \circ X) = \sum_{u \in R^I}f(u) \cdot \Pro(X = u)\).
}
\subsubsubsection{Linearität des EW}{
    Für \(X\) ZV oder Familie davon mit EW und \(\lambda \in \R\) oder gleichlange Familie davon gilt: \(\Ew(\lambda \cdot X) = \lambda \cdot \Ew(X)\) es gilt auch: \(\Ew(X + \lambda) = \Ew(X) + \lambda\)
}
\subsubsubsection{Varianz}{
    Wir definieren die Varianz einer ZV \(X\) als: \(\Var(X) := \Ew((X-\Ew(X))^2)\). Es gilt: \(\Var(\lambda X + a) = \lambda^2 \cdot \Var(X)\).\\
    Die Varianz kann auch so berechnet werden: \(\Var(X) = \Ew(X^2) - \Ew(X)^2\)\\
    Es gilt: \(\Var(\Ber(p)) = p(1-p)\), \(\Var(\Bin(n,p)) = np(1-p)\), \(\Var(\Pois(\lambda)) = \lambda\) und \(\Var(\Geo(p)) = \frac{1-p}{p^2}\)\\
    Für \((X_i)_{i\in I}\) eindliche Familie von ZV, die Paarweise unkorreliert sind und Varianz haben, dann gilt:\\
    \(\Var(\sum_{i \in I}X_i) = \sum_{i \in I}\Var(X_i)\).
}
\subsubsubsection{Kovarianz und Unkorreliertheit}{
    Seien \(X,Y\) ZV mit Varianz. Definiere die Kovarianz \(\Cov(X,Y) := \Ew(X{\cdot}Y) - \Ew(X){\cdot}\Ew(Y)\). \\
    Wir nennen solche ZV unkorreliert, wenn \(\Cov(X,Y) = 0\) gilt. Offenbar folgt dies aus \((X,Y)\) unabhängig.
}
\textsc{\underbar{Abschätzungen}}\\
    Markow-Ungleichung: Sei \(X\) ZV mit EW, sei \(\Pro(X \ge 0)\) = 1 und \(a > 0\). Dann gilt \(\Pro(X \ge a) \le \frac{\Ew(X)}{a}\).\\
    Tschebyschow-Ungleichung: Sei \(X\) ZV mit Varianz und \(a > 0\), dann gilt:\\ \(\Pro(|X-\Ew(X)| \ge a) \le \frac{\Var(X)}{a^2}\) und \(\Pro(X-\Ew(X) \ge a) \le \frac{\Var(X)}{\Var(X) + a^2}\)\\
    Sei \(\mu := \Ew(X), \sigma := \sigma(X), k > 0\), dann gilt nach Tschebyschow:\\ \(\Pro(\mu - k\sigma < X < \mu + k\sigma) = \Pro(|X-\mu| < k\sigma) \ge 1 {-} \frac{1}{k^2}\)
\subsubsection*{Verteilungen mit Dichte}
Sei \((\Omega, \E, \Pro)\) W'Raum. \\ \\
\subsubsubsection{Allgemeine ZV}{
    Wir nennen \(X\QDp \Omega \longrightarrow \R\) eine ZV, wenn gilt: \(\forall u \in \R: [X \le u] \in \E\)
}
\textsc{\underbar{Borellmengen}} \\
    Um sinnvolle Events für eine allgemeine ZV zu konstruieren, definieren wir eine \(\sigma\)-Algebra, die nicht die Potenzmenge von \(\R\) ist. Für \(\mathcal G := \{]{-}\infty, u[ \mid u \in \R\}\), was selbst schon eine gültige \(\sigma\)-Algebra ist, definieren wir \(\mathcal B := \bigcap_{\mathcal G \subseteq \E' \subseteq \mathcal P(\R) \wedge \E \text{ ist \(\sigma\)-Algebra über \(\R\)}}\E'\). \(\mathcal B\) ist \(\sigma\)-Algebra.\\
    Folgendes sind Borellmengen: Intervalle, abzählbare Mengen, Offene Mengen, Vereinigungen abzählbar vieler Borellmengen, Schnitte abzählbar vieler Borellmengen, Komplemente von Borellmengen. \\
    Außerdem ist das kartesische Produkt über Borellmengen wieder Borellmenge, somit übertragen sich auch noch äquivalente Definitionen und Regeln für Folgen von ZV, welche über Folgen von Borellmengen agieren.
    \\ \\
\subsubsubsection{Komb'Satz für allgemeine ZV}{
    Seien \(X,Y\) ZV, \(\lambda \in \R\), \(f\QDp D \longrightarrow \R\) stetig für \(\img(X) \subseteq D \in \mathcal B\).\\
    Dann sind folgendes auch ZV: \(X + \lambda\), \(\lambda X\), \(X + Y\), \(XY\), \(f \circ X\)
}
\subsubsubsection{Verteilung einer allgemeinen ZV}{
    Folgendes W'maß nennen wir die Verteilung einer allgemeinen ZV \(X\):\\
    \(\Pro_X\QDp \mathcal B \longrightarrow [0,1], U \mapsto \Pro(X \in U)\)
}
\subsubsubsection{Kumulative Vertelungsfunktion}{
    Sei \(X\) ZV, dann nennen wir \(F_X\QDp \R \longrightarrow [0,1], u \mapsto \Pro(X{\le}u)\), die kumulative Vertelungsfunktion oder auch CDF von \(X\). Wir nennen \(X\) stetige ZV, wenn \(F_X\) stetig ist. Für die CDF gelten folgende Eigenschaften: \(F_X\) ist monoton steigend, \(F_X\) ist rechtsstetig (also stetig für rechte Grenze), \(\lim_{\xi \rightarrow +\infty}F_X = 1 \wedge \lim_{\xi \rightarrow -\infty}F_X = 0\)\\
    Es gilt außerdem wenn \(F_X\) stetig ist: \(\Pro(X{=}u) = 0\) für alle \(u \in \R\) und \(\Pro(X {\in} I) = \Pro(X {\in} I^o)\) für \(I^o\) das offene Intervall von \(I\).\\
    Man kann eine CDF auch aus der PMF definieren mit: \(\R \longrightarrow [0,1], u \mapsto \sum_{\xi \le u}\rho_X(\xi)\), offenbar sind ZV, die solch eine PMF bzw. CDF haben dann immer diskret, wie wenn die Bildmenge von \(X\) abzählbar ist.
}
\textsc{\underbar{Unabhängigkeit allgemeiner ZV}}\\
    Der Begriff der Unabhängigkeit überträgt sich für allgemeine ZV, nur, dass wir nicht mehr alle Teilmengen sondern nur noch alle Borellmengen betrachten und unendliche ZV-Familien sind erlaubt. \\ \\
\subsubsubsection{Wahrscheinlichkeitsdichtefunktionen}{
    Sei \(\rho\QDp \R \longrightarrow \R_{\ge 0}\) integrierbar über alle kompakten Intervalle, sodass \(\int_{-\infty}^{+\infty}\rho = 1\) gilt, dann nennen wir \(\rho\) PDF oder auch Wahrscheinlichkeitsdichtefunktion.\\
    Ist \(\rho\) PDF, dann ist \(F\QDp \R \longrightarrow [0,1], u \mapsto \int_{-\infty}^{u} \rho\) eine stetige CDF. \\
    Hat die ZV \(X\) eine PDF \(\rho\), dann gilt: \(\Pro(a \le X \le b) = \int_{a}^{b}\rho\). Beachte hier, dass da die Punktevents keine Wahrscheinlichkeit haben, deswegen es egal ist, ob man \(<\) statt \(\le\) verwendet etc.
}
\textsc{\underbar{Erwartungswert via PDF, LOTUS für allgemeine ZV}}\\
    Sei \(X\) allgemeine ZV mit PDF \(\rho\), dann hat \(X\) genau dann einen EW, wenn \(\int_{-\infty}^{+\infty}|t| {\cdot} \rho(t) \text{ d}t < +\infty\) gilt, dann gilt:
    \[\Ew(X) = \int_{-\infty}^{+\infty}t \cdot \rho(t)\text{ d}t\]
    Sei \(f\QDp D \longrightarrow \R\) stetig mit \(\img(X) \subseteq D \subseteq \R\), dann hat \(f \circ X\) gena dann einen EW, wenn \(\int_{-\infty}^{+\infty}|f| {\cdot} \rho < +\infty\) gilt, dann gilt:
    \[\Ew(f \circ X) = \int_{-\infty}^{+\infty}f \cdot \rho\]
\subsubsubsection{Diverse Verteilungen}{
    \textbf{\textit{Uniforme Verteilung}}: Seien \(a < b \in \R\), wir sagen eine ZV \(X\) hat uniforme Verteilung zwischen \(a\) und \(b\) und schreiben \(X \sim \Unif(a,b)\), wenn \(X\) folgende PDF hat:
    \(\rho \QDp \R \longrightarrow \R_{\ge 0}, t \mapsto \frac{1}{b-a}\), wenn \(t \in ]a,b[\) sonst \(0\). Es gilt: \(\Ew(\Unif(a,b)) = \frac{a+b}{2}\) und \(\Var(\Unif(a,b)) = \frac{(b-a)^2}{12}\) \\
    \textbf{\textit{Exponentialverteilung}}: Sei \(\lambda > 0\), wir sagen eine ZV \(X\) hat Exponentialverteilung mit Rate \(\lambda\), wenn folgende Funktion die PDF von \(X\) ist:
    \(\rho \QDp \R \longrightarrow \R_{\ge 0}, t \mapsto \lambda e^{-\lambda t}\), wenn \(t \ge 0\) und sonst \(0\). Es gilt: \(\Ew(\Exp(\lambda)) = \lambda^{-1}\) und \(\Var(\Exp(\lambda)) = \lambda^{-2}\)\\
    \textbf{\textit{Pareto-Verteilung}}: Sei \(a > 0\), wir sagen eine ZV \(X\) hat Pareto-Verteilung, wenn \(X\) folgende PDF hat:
    \(\rho \QDp \R \longrightarrow \R_{\ge 0}, t \mapsto \frac{a}{t^{a+1}}\), wenn \(t \ge 1\) und sonst \(0\). Es gilt: \(\Ew(\text{Par}(a)) = \frac{a}{a-1}\), wenn \(a > 1\) und wenn \(a > 2\) gilt: \(\Var(\text{Par}(a)) = \frac{a}{(a-1)^2(a-2)}\) \\
    \textbf{\textit{Cauchy-Verteilung}}: Sei \(t_0 \in \R, \gamma > 0\), wir sagen \(X\) hat Cauchy-Verteilung mit Lageparameter \(t_0\) und Skalenparameter \(\gamma\), wenn \(X\) folgende PDF hat:
    \(\rho \QDp \R \longrightarrow \R_{\ge 0}, t \mapsto \left(\pi \gamma \left(1+\left(\frac{t-t_0}{\gamma}\right)^2\right)\right)^{-1}\). Die Cauchy-Verteilung hat keinen EW weder Varianz. \\
    \textbf{\textit{Normal-Verteilung}}: Sei \(\mu \in \R\) und \(\sigma > 0\). Wir sagen \(X\) hat Normalverteilung mit EW \(\mu\) und Varianz \(\sigma^2\),\\
    und schreiben \(X \sim \Norm(\mu, \sigma^2)\), wenn \(X\) folgende PDF hat:
    \(\rho \QDp \R \longrightarrow \R_{> 0}, t \mapsto \frac{e^{-\frac{1}{2}\cdot\left(\frac{t {-} \mu}{\sigma}\right)^2}}{\sqrt{2\pi} \cdot \sigma}\)
    }
\subsubsubsection{Eigentschaften der Normalverteilung}{
    Sei \(X \sim \Norm(\mu, \sigma^2)\) und \(\delta \in \R\), \(\lambda \ne 0\). Dann gilt: \(\lambda X + \delta \sim \Norm(\lambda\mu + \delta, \lambda^2\sigma^2)\). Es gilt auch: \(\Ew(X) = \mu\) und \(\Var(X) = \sigma^2\).\\
    Für \((X_i)_{i \in I}\) endliche Familie an ZV mit \(X_i \sim \Norm(\mu_i, \sigma_i^2)\) i.d.d. gilt: \({\sum_{i\in I}} {X_i} \gap \sim \gap {\Norm}{\left(\sum_{i\in I}\mu_i, \sum_{i\in I}\sigma_i^2\right)}\).
}
\subsubsubsection{Gedächtnislosigkeit}{
    Sei \(X\) eine verteilte ZV. Wir nennen die Verteilung gedächtnislos, wenn gilt: \(\forall n,k > 0\QDp \Pro(X\ge n + k \mid X \ge k) = \Pro(X \ge k)\)\\
    Beweis für \(X \sim \Geo(p)\): Es gilt: \( \Pro(X \ge n + k \mid X \ge n) = \frac{\Pro(X \ge n + k \wedge X \ge n)}{\Pro(X \ge n)} = \frac{\Pro(X \ge n + k)}{\Pro(X \ge n)} = \frac{(1-p)^{n+k-1}}{(1^-p)^{n-1}} = (1-p)^k = \Pro(X \ge k)\). \\
    Außerdem ist die Geometrische Verteilung gedächtnislos \gedanke ich bin das auch.
}
\subsubsection*{Konfidenzintervalle in Normalverteilungsmodellen}
\subsubsubsection{Statistisches Modell}{
    \todo
}
\end{document}